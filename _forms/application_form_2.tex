\documentclass[12pt]{oblivoir}
\usepackage{fapapersize,tabu,siunitx, %siunitx는 단위를 식자하기 위한 패키지 \si{...}명령 사용
longtable}
\usepackage[shortlabels]{enumitem} %리스트 환경 정의를 위한 패키지 및 옵션

%A4 용지 및 여백 설정
\usefapapersize{210mm,297mm,20mm,*,45mm,30mm}
% fapapersize 패키지를 사용하지 않을 때 memoir 명령을 사용하여 아래와 같이 설정할 수 있다고 하는데 잘 안됨.
%\setstocksize{297mm}{210mm} %A4 사이즈
%\settrims{0pt}{0pt} %용지와 판면을 일치
%이하는 여백을 먼저 설정하는 방법
%\setlrmarginsandblock{20mm}{20mm}{*} %내측 외측 여백 설정
%\setulmarginsandblock{45mm}{30mm}{*} % 상하 여백 설정
%이하는 typeblock의 크기를 직접 지정하는 방법
%\settypeblocksize{222mm}{170mm}{*}
%\setlrmargins{*}{*}{1.0}
%\setulmargins{*}{*}{1.5}

%%%%%--환경 설정--%%%%%
%1. 리스트 환경 설정
\AddEnumerateCounter*{\gana}{\@gana}{가} %리스트 환경에서 가, 나, 다, 라 ... 목차. enumitem 패키지 전제

%2. 폰트 설정
\setmainfont[WordSpace=2.0]{Source Han Serif} %fontspec 패키지의 WordSpace설정을 사용하기 위해 fontspec 명령어인 \setmainfont 사용. 1행의 글자 수를 45-50자 사이로 맞추기 위한 WordSpace 설정
%\setmainfont[WordSpace=2.0]{HCR Batang}  %폰트를 함초롬체로 설정할 때 사용

%3. Section 양식 설정
\renewcommand{\thesection}{\arabic{section}.} % section 형식을 "1. 제목"의 형식으로 설정
\renewcommand{\thesubsection}{\gana{subsection}.} % subsection 형식을 "가. 제목"의 형식으로 설정

\setsecheadstyle{\normalsize} % section 글자 크기를 본문과 동일하게 설정
\setsubsecheadstyle{\normalsize} % subsection 글자 크기를 본문과 동일하게 설정

\setbeforesecskip{-0ex plus -1ex minus -.2ex} %음수로 설정하면 다음 첫 단락이 들여밀기 되지 않는다. -0는 양수로 인식된다. 따라서 들여밀기를 없애려면 -0.001ex로 설정할 것. subsection에 관해 설정하려면 \setbeforesubsecskip으로 설정. paragraph는 \setbeforeparaskip. 이하 같은 방식 적용.
\setaftersecskip{.001ex plus .001ex} %양수로 설정하면 별행표제
\setsecindent{\parindent} %section title을 들여쓰기

%sectioning division에서 둘째 줄 들여쓰기 되지 않도록 설정. \setbeforesecskip 명령을 사용하면 이 설정은 영향을 미치지 않음
%\makeatletter
%\def\@hangfrom#1{\setbox\@tempboxa\hbox{{#1}}
%  \hangindent 0pt
% \noindent\box\@tempboxa}
%\makeatother

\def\mysection#1{\hfil #1 \hfil} %제목, 신청취지, 신청이유 등을 가운데 정렬하기 위해 \mysection 명령을 정의

%4. 행간, 문단간격 등 설정
\linespread{2.1} % 행간을 한 면에 21줄이 들어가도록 설정

%5. table 환경 설정
% \newcolumntype{b}{>{\hsize=.25\hsize}X} %tabularx의 칼럼 유형. tabularx package를 쓸 때 사용. siunitx package와 함께 쓸 때는 칼럼의 이름을 s 또는 S로 하지 말 것. siunitx package의 S column과 충돌함. tabu package도 마찬가지인데, 다만 sunitx의 S column을 포함해서 재정의할 수는 있음.

% 6. 헤더, 푸터 등 설정
\renewcommand*{\thepage}{--~\arabic{page}~--} %페이지 번호를 - 1 -의 형식으로 표시




\begin{document}
\section*{\mysection{\Large 공사중지 등 가처분 신청서}}
\vspace{5em}\par
\begin{longtabu} to \linewidth{@{}X @{}X[3.5]} %@{}은 셀 안에서 텍스트와 셀 경계 사이의 여백을 없애기 위한 설정
채~~권~~자&\begin{enumerate}[nosep, left=0pt, before=\vspace{-0.35\baselineskip}, after=\vspace{-0.7\baselineskip}] %당사자가 여럿일 때 번호를 부여하기 위한 것. before설정은 list 환경에서 자동으로 삽입되는 행을 제거하여 vertical space를 제거하기 위한 것. left 설정은 list환경에서 왼쪽 여백을 제거하기 위한 것
\item 김유비(710346-1817262)\\서울 서초구 신반포로 23(반포동)\\소송대리인 변호사 박소신\\서울 서초구 서초동 4321 승리빌딩 101호\\\mbox{전화 02-532-2233, 팩스 02-533-1234, 이메일 anwalt@legaltalks.co.kr}
\item 박장비(791231-1234567)\\고양시 일산동구 킨텍스로 23
\end{enumerate}\\
채~~무~~자&이관우(800524-6543121)\\
&경남 고성군 엑스포로 35\\
목적물의 표시&별지 목록 기재와 같음\\
소송목적의 값&ㅇㅇㅇ,ㅇㅇㅇ,ㅇㅇㅇ원\\
&서울 서초구 신반포로 23 대 112의 개별공시지가 123,456,789원$\times\frac{1}{2}\times\frac{1}{2}$\\
피보전권리의 내용&소유권에 기한 방해예방청구권 및 방해배제청구권
\end{longtabu}
\par
\vspace{5em}
\section*{\mysection{\large 신청취지}}
\vspace{1em}
\begin{enumerate}[nosep, left=0pt] %item 사이의 간격을 linespace와 같게 하고, item label을 들여쓰지 않고 시작하는 옵션. enumitem 전제
\item 채무자는 별지 목록 기재 제1토지에서 같은 목록 기재 제2토지와의 경계선에 높이 9m, 길이 40m 이상의 철근콘크리트조 옹벽을 설치하지 아니하고 터파기 공사를 하여서는 아니 된다.
\item 채무자는 위 목록 기재 제1토지에서 17:00부터 다음날 9:00까지 사이에 일체의 공사를 하여서는 아니 된다.
\item 채무자는 위 목록 기재 제1토지에서 제2항 기재 시간 이외의 시간 중에 55\si{\decibel}을 초과하는 소음을 내어서는 아니 된다.
\item 집행관은 제1항 내지 제3항의 취지를 적당한 방법으로 공시하여야 한다.
\end{enumerate}
라는 결정을 구합니다.
\vspace{1em}
\section*{\mysection{\large 신청이유}}
\vspace{1em}
\section{방해예방청구 가처분 신청 부분}
가. 채권자는 서울 서초구 신반포로 23 대 112\si{\square\metre}(이하 `이 사건 제2토지'라 합니다) 및 그 지상 시멘트 블록조 슬라브지붕 1층 단독주택 78\si{\square\metre}의 소유자로서 2000년경부터 그 지상 주택에 가족들과 함께 거주하고 있고, 채무자는 그와 경계선이 접한 같은 도로 24 대 112\si{\square\metre}(이하 `이 사건 제1토지'라 합니다)의 소유자로서 2019. 4. 1.부터 나대지인 그 지상에 지하 2층 지상 5층의 상가건물을 신축하고 있는 건축주입니다.\par
나. 그런데 2019. 4. 15. 본격적인 터파기 공사가 시작된 이후인 같은 달 20일 경 이 사건 제1토지와 연접한 채권자 소유의 1.5\si{\metre} 높이의 블록조 담벽(2010년경 축조)에 지름 약 10\si{\milli\metre}, 길이 약 70\si{\centi\metre} 크기의 균열이 발생하였습니다. 이 사건 공사는 지하 2층까지 예정되어 있어서 앞으로도 최소 7$\sim$8\si{\metre}는 더 굴착이 필요한데, 지반침하 내지 사면 붕괴 등의 위험에 대한 보강공사 없이 이 상태로 공사가 계속 진행되면 채권자 소유의 이 사건 제2토지 지상 주택에 심각한 침해를 가져올 개연성이 있습니다.\par
이러한 손해가 이론적으로는 사후에 금전으로 전보될 수 있다고 하더라도 이와 같이 현저한 손해의 발생이 예견된다면 미리 조치를 취하도록 함으로써 사전에 손해의 발생을 예방하는 것이 타당한 분쟁해결의 방법입니다.\par
채무자는 채권자에게 공사 완료 후 담벽을 새로 설치해주겠다고 제안하였는데, 그것은 그가 당연히 해야 할 내용이고, 채권자는 앞으로 자신의 주택에 어떠한 안전상의 위해가 발생할지에 대해 매우 불안해 하는 것입니다.\par
다. 따라서 채무자가 예정 굴착깊이인 9\si{\metre}로 이 사건 제1, 2토지의 경계선(약 40\si{\metre})에 걸쳐 철근콘크리트 옹벽을 설치할 때까지는 터파기 공사를 못 하도록 할 보전의 필요성이 있습니다.
\section{방해배제청구 가처분 신청 부분}
채무자는 이 사건 제1토지에서 6$\sim$7월 장마철을 피해 완공할 목적으로 터파기 공사를 24시간 철야로 진행하고 있습니다. 그로 인해 공사현장 바로 이웃에서 살고 있는 채권자는 수면, 일상적 대화 등에 매우 심각한 지장을 받고 있습니다. 또한 채권자가 1,000만 원을 들여 구입한 오디오 시설도 무용지물이 되었습니다.\par
채권자가 매일 공사현장을 방문하여 인부들에게 공사중단을 요청하고, 채무자에게도 중단을 요청하였으나 채무자는 이를 거절하였습니다. 급기야 채권자가 공사현장의 차량 통행로를 막았으나 채무자가 공사방해로 신고하여 현재도 24시간 철야로 터파기 공사가 진행되고 있고 채무자는 공사를 중단할 의사가 없음이 명백합니다.\par
한편 이 사건 제1, 2토지 주위는 종래 나대지였고, 현재는 주택들이 밀집하여 있는 곳입니다. 그렇다면 통상적으로 건설공사가 종료하는 시각, 주거지역이라는 사정 등을 고려하여 위 신청취지 제2항 및 제3항과 같이 17:00부터 다음날 09:00까지는 일체의 공사를 금지하고, 그 외의 시간에도 환경정책기본법 시행령 제2조(환경기준) 별표 2. 소음 중 국토의 계획 및 이용에 관한 법률 시행령 제30조 제1호 나목 및 다목에 따른 일반주거지역 및 준주거지역의 낮시간 동안에 적용되는 55\si{\decibel}을 초과하는 소음을 발생시키는 공사를 금지할 필요가 있습니다.
\section{공시의 신청}
신청취지 제1항 내지 제3항은 그 실효성을 확보하기 위해 이 사건 공사의 수급인 또는 현장인부들에게도 알려져야 하나, 채권자가 이 사건 공사의 수급인 또는 현장인부들에 대한 인적 사항을 전혀 알지 못하고 알 방법이 없으므로 집행관으로 하여금 적당한 방법으로 공시하게 하여 주시기 바랍니다.
\section{담보의 제공}
담보의 제공은 보증보험회사와 지급보증위탁계약을 체결한 각 문서를 각 제출하는 것으로 갈음하는 것을 허가하여 주시기 바랍니다.
\vspace{5em}
\section*{\mysection{소명방법}}
\vspace{2em}
\begin{tabu} to \linewidth{X}
    1. 소갑 제1호증의 1 내지 3(각 등기사항전부증명서)\\
    1. 소갑 제2호증(건축허가)\\
    1. 소갑 제3호증(내용증명)\\
    1. 소갑 제4호증(사진)\\
    1. 소갑 제5호증의 1 내지 9(각 사진)\\
    1. 소갑 제6호증(사진)\\
    1. 소갑 제7호증의 1 내지 4(각 사진)\\
    1. 소갑 제8호증(박준모의 확인서)
  \end{tabu}
 % \enlargethispage{\baselineskip}
\bigskip
\section*{\mysection{첨부서류}}
\vspace{2em}
\begin{tabu}to \linewidth{X X}
  1. 각 소명방법&1통\\
  1. 공탁보증보험증권&각1통\\
  1. 송달료 납부서&1통\\
  1. 소송위임장&1통
\end{tabu}
\par\vspace{5em}
\centering\today{}\par
\vspace{\stretch{1}}
\raggedleft 채권자 소송대리인 변호사 박소신\\
\vspace{\stretch{2}}
\raggedright 서울중앙지방법원 귀중
\par





%여기서부터 별지
\newpage
\pagenumbering{gobble} %page number 생략. page number를 다시 시작하려면 \pagenumbering{arabic} 사용. arabic 이외에 alph, Alph, roman, Roman 등등 사용 가능.
별지\\
\bigskip
\section*{\mysection{목~~~~~록}}
\begin{enumerate}
\item 서울 서초구 신반포로 24 대 112\si{\square\metre}\\
\item 서울 서초구 신반포로 23 대 112\si{\square\metre}. 끝.\\
% \item (1동의 건물의 표시)\\
%  서울 서초구 잠원동 12 공주아파트 제102동\\
% { [도로명주소]} 서울 서초구 잠원로 43\\
 % 철근콘크리트조 박공지붕 15층 아파트 지하1층 ㅇㅇ\si{\square\metre}, 1층 ㅇ,ㅇㅇㅇ\si{\square\metre}, 2층 ㅇ,ㅇㅇㅇ\si{\square\metre}, 3층 ㅇ,ㅇㅇㅇ\si{\square\metre}, 4층 ㅇ,ㅇㅇㅇ\si{\square\metre}, 5층 ㅇ,ㅇㅇㅇ\si{\square\metre}, 6층 ㅇ,ㅇㅇㅇ\si{\square\metre}, 7층 ㅇ,ㅇㅇㅇ\si{\square\metre}, 8층 ㅇ,ㅇㅇㅇ\si{\square\metre}, 9층 ㅇ,ㅇㅇㅇ\si{\square\metre}, 10층 ㅇ,ㅇㅇㅇ\si{\square\metre}, 11층 ㅇ,ㅇㅇㅇ\si{\square\metre}, 12층 ㅇ,ㅇㅇㅇ\si{\square\metre}, 13층 ㅇ,ㅇㅇㅇ\si{\square\metre}, 14층 ㅇ,ㅇㅇㅇ\si{\square\metre}, 15층 ㅇ,ㅇㅇㅇ\si{\square\metre}\\
%  (대지권의 목적인 토지의 표시)\\
%  서울 서초구 잠원동 12 대 ㅇㅇ,ㅇㅇㅇ\si{\square\metre}\\
%  (전유부분의 건물의 표시)\\
%  제2층 제201호 철근콘크리트조 ㅇㅇㅇ\si{\square\metre}\\
%  (대지권의 표시)\\
%  소유권 대지권 ㅇㅇ,ㅇㅇㅇ분의 ㅇㅇ. 끝.
\end{enumerate}
\end{document}
