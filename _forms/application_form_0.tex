\documentclass[12pt]{oblivoir}
\usepackage{fapapersize,tabu,siunitx, %siunitx는 단위를 식자하기 위한 패키지 \si{...}명령 사용
longtable}
\usepackage[shortlabels]{enumitem} %리스트 환경 정의를 위한 패키지 및 옵션

%%%%%A4 용지 및 여백 설정%%%%%
\usefapapersize{210mm,297mm,20mm,*,45mm,30mm}
% fapapersize 패키지를 사용하지 않을 때 memoir 명령을 사용하여 아래와 같이 설정할 수 있다고 하는데 잘 안됨.
%\setstocksize{297mm}{210mm} %A4 사이즈
%\settrims{0pt}{0pt} %용지와 판면을 일치
%이하는 여백을 먼저 설정하는 방법
%\setlrmarginsandblock{20mm}{20mm}{*} %내측 외측 여백 설정
%\setulmarginsandblock{45mm}{30mm}{*} % 상하 여백 설정
%이하는 typeblock의 크기를 직접 지정하는 방법
%\settypeblocksize{222mm}{170mm}{*}
%\setlrmargins{*}{*}{1.0}
%\setulmargins{*}{*}{1.5}

%%%%%--환경 설정--%%%%%
%%1. 리스트 환경 설정
\AddEnumerateCounter*{\gana}{\@gana}{가} %리스트 환경에서 가, 나, 다, 라 ... 목차. enumitem 패키지 전제

%%2. 폰트 설정
\setmainfont[WordSpace=2.0]{Source Han Serif} %fontspec 패키지의 WordSpace설정을 사용하기 위해 fontspec 명령어인 \setmainfont 사용. 1행의 글자 수를 45-50자 사이로 맞추기 위한 WordSpace 설정
% \setmainfont[WordSpace=2.0]{HCR Batang}  %폰트를 함초롬체로 설정할 때 사용
\setsansfont[WordSpace=2.0]{Noto Sans CJK KR}

%%3. Section 양식 설정
%3.2 명령을 재정의한 부분
\renewcommand{\chaptitlefont}{\sffamily\large\centering} %Sans Serif체로 가운데 정렬. 번호 없는 목차를 위한 것
\renewcommand{\printchapternonum}{} %chapter number에 관하여 아무 작업을 하지 않도록 한 명령
\renewcommand{\thesection}{\arabic{section}.} % section 형식을 "1. 제목"의 형식으로 설정
\renewcommand{\thesubsection}{\gana{subsection}.} % subsection 형식을 "가. 제목"의 형식으로 설정
%3.2 section head style 설정
\setsecheadstyle{\normalsize} % section 글자 크기를 본문과 동일하게 설정
\setsubsecheadstyle{\normalsize} % subsection 글자 크기를 본문과 동일하게 설정
%3.3 chapter 관련 설정
\setlength{\beforechapskip}{-2\baselineskip} %chapter 목차의 시작하는 지점을 조정. 문서의 타이틀에서 윗 공백이 많이 생기는 것을 줄이기 위함
\setlength{\midchapskip}{0pt} %chapter의 번호가 있는 위치와 chapter의 타이틀이 오는 수직 간격을 제거
\setlength{\afterchapskip}{1em} %chapter 이후의 수직 간격을 제거
%3.4 들여쓰기 여부 설정
\setbeforesecskip{-0ex plus -1ex minus -.2ex} %음수로 설정하면 다음 첫 단락이 들여밀기 되지 않는다. -0는 양수로 인식된다. 따라서 들여밀기를 없애려면 -0.001ex로 설정할 것. subsection에 관해 설정하려면 \setbeforesubsecskip으로 설정. paragraph는 \setbeforeparaskip. 이하 같은 방식 적용.
\setaftersecskip{.001ex plus .001ex} %양수로 설정하면 별행표제
\setsecindent{\parindent} %section title을 들여쓰기

%sectioning division에서 둘째 줄 들여쓰기 되지 않도록 설정. \setbeforesecskip 명령을 사용하면 이 설정은 영향을 미치지 않음
%\makeatletter
%\def\@hangfrom#1{\setbox\@tempboxa\hbox{{#1}}
%  \hangindent 0pt
% \noindent\box\@tempboxa}
%\makeatother

%\def\mysection#1{\hfil #1 \hfil} %제목, 신청취지, 신청이유 등을 가운데 정렬하기 위해 \mysection 명령을 정의

%%4. 행간, 문단간격 등 설정
\linespread{2.1} % 행간을 한 면에 21줄이 들어가도록 설정

%%5. table 환경 설정
% \newcolumntype{b}{>{\hsize=.25\hsize}X} %tabularx의 칼럼 유형. tabularx package를 쓸 때 사용. siunitx package와 함께 쓸 때는 칼럼의 이름을 s 또는 S로 하지 말 것. siunitx package의 S column과 충돌함. tabu package도 마찬가지인데, 다만 sunitx의 S column을 포함해서 재정의할 수는 있음.

%%6. 헤더, 푸터 등 설정
\renewcommand*{\thepage}{--~\arabic{page}~--} %페이지 번호를 - 1 -의 형식으로 표시

%%%%%원문자 사용을 위한 명령 설정(tikz 패키지 필요)%%%%%%
\newcommand*\circled[1]{\tikz[baseline=(char.base)]{
     \node[shape=circle,draw,inner sep=0.5pt] (char) {#1};}}




%%%%%%%%%% 문서 시작 부분%%%%%%%%%%%

\begin{document}
%%%%%% 설정에 관한 언급%%%%%%
% 1. 문서의 제목은 \chapter*{\Large } 명령을 사용할 것
% 2. 번호가 붙지 않는 제목은 \chapter* 명령을 사용할 것. 다만, 증명(소명)방법, 첨부서류, 목록 등은 \chapter*{\normalfont\normalsize } 명령을 사용할 것
% 3. 당사자가 수인인 경우 enumerate 환경을 사용할 것. 만약 주석 처리되어 있으면 주석을 제거하고 &를 주석 처리하면 됨.
%     반대의 경우는 &의 주석을 제거하고 enumerate 환경을 주석 처리하면 됨
%%%%%%%%%%%%%%%%%%%%

\chapter*{\Large 소~~~~장}
\begin{longtabu} to \linewidth{@{}X @{}X[3.5]} %@{}은 셀 안에서 텍스트와 셀 경계 사이의 여백을 없애기 위한 설정
 원~~~~고&
 \begin{enumerate}[nosep, left=0pt, before=\vspace{-0.35\baselineskip}, after=\vspace{-0.7\baselineskip}] %당사자가 여럿일 때 번호를 부여하기 위한 것. before설정은 list 환경에서 자동으로 삽입되는 행을 제거하여 vertical space를 제거하기 위한 것. left 설정은 list환경에서 왼쪽 여백을 제거하기 위한 것
\item
  김유비(710346-1817262)\\
 % &
  서울 서초구 신반포로 23(반포동)\\
  %&
  소송대리인 변호사 박소신\\
  %&
  서울 서초구 서초동 4321 승리빌딩 101호\\
  %&
  \mbox{전화 02-532-2233, 팩스 02-533-1234, 이메일 bictory@naver.com}
\item
박장비(791231-1234567)\\고양시 일산동구 킨텍스로 23
\end{enumerate}
  \\
피~~~~고&조운장\\
&서울 서초구 잠원로 43, 102동 201호(잠원동, 공주아파트)\\
\end{longtabu}
\vspace{2em}
대여금 등 청구의 소
\vspace{3em}
\chapter*{\large 청구취지}
\begin{enumerate}[nosep, left=0pt] %item 사이의 간격을 linespace와 같게 하고, item label을 들여쓰지 않고 시작하는 옵션. enumitem 전제
\item 피고는 원고들에게 각 100,000,000원 및 이에 대한 2019. 1. 1.부터 소장 부본 송달일까지는 연 5\%의, 그 다음날부터 다 갚는 날까지는 연 12%의 각 비율로 계산한 각 돈을 지급하라.
\item 소송비용은 피고의 부담으로 한다.
\item 제1항은 가집행할 수 있다.
\end{enumerate}
라는 판결을 구합니다.
\vspace{5em}
\chapter*{청구원인}
원고 김유비는 \ldots
\vspace{5em}
\chapter*{\normalfont\normalsize 증명방법}
\begin{longtabu} to \linewidth{X}
  1. 갑 제1호증(대여금 계약서)\\
  1. 갑 제2호증(계좌이체 내역서)\\
  1. 갑 제3호증(영수증)
\end{longtabu}
\vspace{5em}
\chapter*{\normalfont\normalsize 첨부서류}
\begin{longtabu} to \linewidth{X X}
  1. 각 증명방법&1통\\
  1. 송달료 납부서&1통\\
  1. 소송위임장&1통
\end{longtabu}
\vspace{5em}
\centering\today{}\par
\vspace{\stretch{1}}
\raggedleft 원고 소송대리인 변호사 최은석\\
\vspace{\stretch{2}}
\raggedright ㅇㅇ지방법원 귀중





% 여기서부터 별지
\newpage
\pagenumbering{gobble}
별지
\vspace{5em}
\chapter*{\normalfont\normalsize 목~~~~록}
\begin{enumerate}
\item 서울 서초구 반포동 24 대 112\si{\square\metre}\\ %토지를 특정하는 경우
\item 광주시 ㅇㅇ동 112(수인로2길 55) 지상 벽돌조 박공지붕 2층 식당 1층 98\si{\square\metre} 2층 30\si{\metre\squared}\\ %건물을 특정하는 경우
\item 대전 유성구 대학로 98 잡종지 300\si{\square\metre} 지상 철근콘크리트조 지하 1층 지상 5충 사무실 지하 1층 200\si{\square\metre}, 지상 1층 200\si{\square\metre}, 2층 200\si{\metre\squared}, 3층 200\si{\square\metre}, 4층 200\si{\square\metre}, 5층 120\si{\metre\squared}, 옥탑 20\si{\square\metre}\\ %토지와 건물을 동시에 특정하는 경우
\item (1동의 건물의 표시)\\ %공동주택을 특정하는 경우
  충남 서천군 마서면 덕암리 333 천산아파트 1동\\
  {[도로명주소]}충남 서천군 장서로 21\\
  철근콘크리트 기와지붕 15층 아파트1층 ㅇ,ㅇㅇㅇ\si{\square\metre}, 2층 ㅇ,ㅇㅇㅇ\si{\square\metre}, 3층 ㅇ,ㅇㅇㅇ\si{\square\metre}, 4층 ㅇ,ㅇㅇㅇ\si{\square\metre}, 5층 ㅇ,ㅇㅇㅇ\si{\square\metre}, 6층 ㅇ,ㅇㅇㅇ\si{\square\metre}, 7층 ㅇ,ㅇㅇㅇ\si{\square\metre}, 8층 ㅇ,ㅇㅇㅇ\si{\square\metre}, 9층 ㅇ,ㅇㅇㅇ\si{\square\metre}, 10층 ㅇ,ㅇㅇㅇ\si{\square\metre}, 11층 ㅇ,ㅇㅇㅇ\si{\square\metre}, 12층 ㅇ,ㅇㅇㅇ\si{\square\metre}, 13층 ㅇ,ㅇㅇㅇ\si{\square\metre}, 14층 ㅇ,ㅇㅇㅇ\si{\square\metre}, 15층 ㅇ,ㅇㅇㅇ\si{\square\metre}\\
  (대지권의 목적인 토지의 표시)\\
  충남 서천군 마서면 덕암리 333 대 2,000\si{\square\metre}, 334 임야\si{\square\metre}\\
  (전유부분의 건물의 표시)\\
  제7층 제704호 철근콘크리트조 98\si{\square\metre}\\
  (대지권의 표시)\\
  소유권 대지권 2,334분의 80.
\item 서울 강남구 삼성동 37-18(봉은사로 408) 지상 벽돌조 기와지붕 단층 주택 90\si{\square\metre}{[등기기록상 표시 : 같은 동 37-18(봉은사로 408) 지상 시멘트벽돌조 기와지붕 단층 주택 80\si{\square\metre}]}. %건물의 현황과 등기기록이 다른 경우 표시 방법. 건물은 현황을 기준으로 등기기록을 병기
  \item 서울 강남구 삼성동 37-18 대 1,500\si{\metre\squared}{[등기기록상 표시 : 같은 동 37-18 임야 1,300평]}. 끝. %토지의 토지대장과 등기기록이 다른 경우 표시 방법. 토지는 토지대장을 기준으로 등기기록을 병기
\end{enumerate}
\end{document}
